\documentclass[11pt, a4paper, sans]{moderncv}

\moderncvstyle{classic}
\moderncvcolor{blue}
\moderncvicons{awesome}

\usepackage[ru, en]{variants}

\usepackage{cmap}
\usepackage[T1, T2A]{fontenc}
%\usepackage[utf8]{inputenc}
\usepackage[russian, english]{babel}
\usepackage[default]{droidserif}
\usepackage[defaultsans]{droidsans}

\usepackage[left=2cm, right=2cm, top=2cm, bottom=2.25cm]{geometry}
\setlength{\hintscolumnwidth}{2cm}

\usepackage{xcolor}

\usepackage{gitinfo2}

% Patches
\usepackage{xpatch}
\xpatchcmd{\makecvhead}{\\[2.5em]}{\\}{}{}
\renewcommand*{\mobilephonesymbol}{{\small\faMobile*}~}
\renewcommand*{\emailsymbol}{{\small\faEnvelope[regular]}~}

\AtEndPreamble{
\hypersetup{
    pdfsubject  = {\ru{Резюме}\en{Resume}},
    pdfkeywords = {cv, resume, curriculum vit\ae{}, r\'{e}sum\'{e}, резюме}
}
}

%%%%%%%%%%%%%%%%%%%%%%%%%%%%%%%%%%%%%%%%%%%%%%%%%%%%%%%%%%%%%%%%%%%%%%%%%%%%%%%%
%                                    TITLE                                     %
%%%%%%%%%%%%%%%%%%%%%%%%%%%%%%%%%%%%%%%%%%%%%%%%%%%%%%%%%%%%%%%%%%%%%%%%%%%%%%%%

\name{%
    \ru{Егор}\en{Egor}%
}{%
    \ru{Макаренко}\en{Makarenko}%
}

\title{%
    C++~/~Python \ru{разработчик}\en{software developer}%
}

\phone[mobile]{+7~(999)~203~83~35}
\email{egormkn@yandex.ru}
\social[github]{egormkn}
\social[telegram]{egormkn}
\social[linkedin]{egormkn}
% \extrainfo{Some info}
% \photo[64pt][0.5pt]{picture}

%%%%%%%%%%%%%%%%%%%%%%%%%%%%%%%%%%%%%%%%%%%%%%%%%%%%%%%%%%%%%%%%%%%%%%%%%%%%%%%%
%                                    CONTENT                                   %
%%%%%%%%%%%%%%%%%%%%%%%%%%%%%%%%%%%%%%%%%%%%%%%%%%%%%%%%%%%%%%%%%%%%%%%%%%%%%%%%

\begin{document}
\makecvtitle

\section{%
  \ru{Образование}\en{Education}%
 }

\cventry{2015--2019}{%
  \ru{Бакалавриат, Прикладная математика и информатика}%
  \en{BSc, Applied Mathematics and Computer Science}%
}{%
  \ru{\newline факультет информационных технологий и программирования, \newline Университет ИТМО}%
  \en{Information Technologies and Programming Faculty, ITMO University}%
}{%
  \ru{Санкт-Петербург}\en{Saint Petersburg}%
}{}{%
  \ru{Профиль: математические модели и алгоритмы в разработке ПО. \\ Тема ВКР: Решатель для задачи анализа вариации потока с термодинамическими ограничениями на основе сведения к целочисленному линейному программированию.}%
  % TODO: English variant
}

\cventry{\ru{с}\en{since} 2019}{%
  \ru{Магистратура, Прикладная математика и информатика}%
  \en{MSc, Applied Mathematics and Computer Science}%
}{%
  \ru{\newline факультет информационных технологий и программирования, \newline Университет ИТМО}%
  \en{Computer Technologies department, ITMO University}%
}{%
  \ru{Санкт-Петербург}\en{Saint Petersburg}%
}{}{%
  \ru{Профиль: программирование и искусственный интеллект.}%
  % TODO: English variant
}

\subsection{%
  \ru{Курсы и профессиональная литература}%
  \en{Courses \& books}%
}

\cvitem{Coursera:}{%
  \ru{Специализация <<\textbf{Искусство разработки на современном C++}>> --- Яндекс, МФТИ}%
  \en{Specialization "\textbf{C++ Modern Development}" --- Yandex, MIPT}%
}

\cvitem{\ru{Книги}\en{Books}:}{%
  \ru{Архитектура компьютера --- Э. Таненбаум}%
  \en{Structured Computer Organization --- A. Tanenbaum}%
}

\section{%
  \ru{Опыт работы}%
  \en{Work experience}%
 }
\cventry{2019--2020}{%
  \ru{Стажёр}\en{Intern}%
}{%
  \ru{Международная лаборатория <<Компьютерные технологии>>, Университет ИТМО}%
  \en{International Laboratory "Computer Technologies", ITMO University}%
}{}{}{%
  % TODO: General description no longer than 1--2 lines.
}

\section{%
  \ru{Профессиональные навыки}%
  \en{Professional skills}%
 }\closesection

\begin{itemize}
  \item Achievement 1
  \item Achievement 2 (with sub-achievements)
        \begin{itemize}
          \item Sub-achievement (a);
          \item Sub-achievement (b), with sub-sub-achievements (don't do this!);
                \begin{itemize}
                  \item Sub-sub-achievement i;
                  \item Sub-sub-achievement ii;
                  \item Sub-sub-achievement iii;
                \end{itemize}
          \item Sub-achievement (c);
        \end{itemize}
  \item Achievement 3
  \item Achievement 4
\end{itemize}

\section{%
  \ru{Проекты}\en{Projects}%
 }\closesection

Your content here, inside the normal \LaTeX environment. You can use any regular \LaTeX{} command, display mathematics, put some table or figure, \dots

\[e =m\,c^2,\]

\emptysection{}
\cvline{Now}{Back to moderncv layout, without making a new section :-)}
\cvline{Long}{Lorem ipsum dolor sit amet, consectetur adipiscing elit. Donec convallis suscipit turpis, et suscipit tortor malesuada non. Sed sit amet dignissim ex. Pellentesque in ipsum at nunc faucibus iaculis ut ac justo. Vestibulum accumsan neque lectus, nec consectetur mauris tincidunt ut.}

\cfoot{\small\textcolor{gray}{\href{https://github.com/egormkn/cv/releases}{https://github.com/egormkn/cv | \gitFirstTagDescribe}}}

\end{document}
