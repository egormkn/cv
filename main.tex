\PassOptionsToPackage{fixed}{fontawesome5}

\documentclass[11pt, a4paper, sans, nolmodern]{moderncv}

\moderncvstyle{classic}
\moderncvcolor{blue}
\moderncvicons{awesome-color}

\usepackage[ru, en]{variants}

%\usepackage[resetfonts]{cmap}
\usepackage[T1, T2A]{fontenc}
%\usepackage[utf8]{inputenc}
\usepackage[russian, english]{babel}

\usepackage[default]{droidserif}
\usepackage[defaultsans]{droidsans}

\usepackage[left=2cm, right=2cm, top=2cm, bottom=2.25cm]{geometry}
\setlength{\hintscolumnwidth}{2cm}

\usepackage{gitinfo2}
\usepackage{selectiontext}

% Patches
%\usepackage{xpatch}
%\xpatchcmd{\makecvhead}{\\[2.5em]}{\\}{}{}

\AtEndPreamble{
\hypersetup{
    pdfsubject  = {\ru{Резюме}\en{Resume}},
    pdfkeywords = {cv, resume, curriculum vit\ae{}, r\'{e}sum\'{e}, резюме}
}
}

%%%%%%%%%%%%%%%%%%%%%%%%%%%%%%%%%%%%%%%%%%%%%%%%%%%%%%%%%%%%%%%%%%%%%%%%%%%%%%%%
%                                    TITLE                                     %
%%%%%%%%%%%%%%%%%%%%%%%%%%%%%%%%%%%%%%%%%%%%%%%%%%%%%%%%%%%%%%%%%%%%%%%%%%%%%%%%

\name{%
    \ru{Егор}\en{Egor}%
}{%
    \ru{Макаренко}\en{Makarenko}%
}

\title{%
    C++~/~Python \ru{разработчик}\en{software developer}%
}

\phone[mobile]{+7~(999)~203~83~35}
\email{egormkn@yandex.ru}
\social[github]{egormkn}
\social[telegram]{egormkn}
\social[linkedin]{egormkn}
% \extrainfo{Some info}
\photo[58pt][0.2pt]{photo}

%%%%%%%%%%%%%%%%%%%%%%%%%%%%%%%%%%%%%%%%%%%%%%%%%%%%%%%%%%%%%%%%%%%%%%%%%%%%%%%%
%                                    CONTENT                                   %
%%%%%%%%%%%%%%%%%%%%%%%%%%%%%%%%%%%%%%%%%%%%%%%%%%%%%%%%%%%%%%%%%%%%%%%%%%%%%%%%

\begin{document}
\makecvtitle

\section{%
  \ru{Образование}\en{Education}%
 }

\cventry{\ru{с}\en{since} 2019}{%
  \ru{Магистратура, Прикладная математика и информатика}%
  \en{MSc, Applied Mathematics and Computer Science}%
}{%
  \ru{\newline факультет информационных технологий и программирования, \newline Университет ИТМО}%
  \en{\newline Information Technologies and Programming Faculty, ITMO University}%
}{%
  \ru{Санкт-Петербург}\en{\newline Saint Petersburg}%
}{}{%
  \ru{Профиль: программирование и искусственный интеллект.}%
  \en{Programming and Artificial Intelligence}%
}

\cventry{2015--2019}{%
  \ru{Бакалавриат, Прикладная математика и информатика}%
  \en{BSc, Applied Mathematics and Computer Science}%
}{%
  \ru{\newline факультет информационных технологий и программирования, \newline Университет ИТМО}%
  \en{\newline Information Technologies and Programming Faculty, ITMO University}%
}{%
  \ru{Санкт-Петербург}\en{\newline Saint Petersburg}%
}{}{%
  \ru{Профиль: математические модели и алгоритмы в разработке ПО. \\ Тема ВКР: Решатель для задачи анализа вариации потока с термодинамическими ограничениями на основе сведения к целочисленному линейному программированию.}%
  \en{Mathematical Models and Algorithms in Software Development \\ Thesis: Solver for the flux variance analysis problem with thermodynamical constraints based on reduction to mixed integer linear programming.}%
}

\subsection{%
  \ru{Дополнительные курсы}%
  \en{Additional courses}%
}

\cvitem{Coursera:}{%
  \ru{Специализация <<\textbf{Искусство разработки на современном C++}>> --- Яндекс, МФТИ}%
  \en{Specialization "\textbf{Modern C++ Development}" --- Yandex, MIPT}%
}

\cvitem{SPTDC:}{%
  Summer school on practice and theory of distributed computing, 2019 --- JUG Ru%
}

% \cvitem{\ru{Книги}\en{Books}:}{%
%   \ru{Архитектура компьютера --- Э. Таненбаум}%
%   \en{Structured Computer Organization --- A. Tanenbaum}%
% }

\section{%
  \ru{Опыт работы}%
  \en{Work experience}%
 }

\cventry{2019--2020}{%
  \ru{Стажёр}\en{Intern}%
}{%
  \ru{Международная лаборатория <<Компьютерные технологии>>, \newline Университет ИТМО}%
  \en{International Laboratory "Computer Technologies", ITMO University}%
}{}{}{%
  \ru{Улучшение инструментов анализа и визуализации данных.}%
  \en{Improving data analysis and visualization tools.}%
}

\cventry{2016--2019}{%
  \ru{Волонтёр}\en{Volunteer}%
}{%
  \ru{Командные олимпиады по программированию (NEERC, ВКОШП), Университет ИТМО}%
  \en{Northeastern European Regional Contest, ITMO University}%
}{}{}{}

\section{%
  \ru{Профессиональные навыки}%
  \en{Professional skills}%
 }

\cvitem{\textcolor{color1}{\selectiontext{\ru{Языки программирования}\en{Programming languages}: }{\faCode}}}{%
  \textbf{C++} (STL, Qt), \textbf{Python} (NumPy, Tensorflow), Java, Javascript, Typescript, Bash; \newline \ru{также знаком с}\en{also familiar with} x86 Assembly, Haskell, SQL.
}
\cvitem{\textcolor{color1}{\selectiontext{\ru{Инструменты}\en{Tools}: }{\faTools}}}{%
  Linux, Git, Make, CMake, Docker, HTML, CSS, Markdown, LaTeX
}
\cvitem{\textcolor{color1}{\selectiontext{\ru{Языки}\en{Languages}: }{\faGlobe}}}{%
  \ru{Английский язык (уровень C1 Advanced)}%
  \en{Russian (Native), English (C1 Advanced)}%
}
\cvitem{\textcolor{color1}{\selectiontext{\ru{Знания}\en{Knowledge}: }{\faLightbulb}}}{%
  \ru{Алгоритмы и структуры данных, многопоточное программирование, основы машинного обучения.}%
  \en{Algorithms and data structures, multithreading, basics of machine learning.}%
}
\cvitem{\textcolor{color1}{\selectiontext{\ru{Опыт}\en{Experience}: }{\faSitemap}}}{%
  \ru{Есть опыт разработки Android приложений, бэкенда на NestJS, обучения нейронных сетей, программирования для Arduino, создания расширений для браузера и VSCode.}%
  \en{I have an experience in developing Android applications, backend development with NestJS framework, training neural networks, working with Arduino, developing extensions for web browsers and VSCode.}%
}

\section{%
  \ru{Проекты}\en{Projects}%
 }\closesection

\ru{Мои личные проекты, курсовые работы и решения задач можно найти на \href{https://github.com/egormkn/}{\textbf{github.com/egormkn}}:
  \\
  \begin{itemize}
    \setlength{\itemindent}{2em}
    \item \href{https://github.com/egormkn/mbr-boot-manager}{\color{color1} mbr-boot-manager} --- загрузочный сектор для x86 c выбором раздела,
    \item \href{https://github.com/egormkn/sdlxx}{\color{color1} SDLXX} --- С++ обёртка для C-библиотеки SDL с простым менеджером 2D-сцен,
    \item \href{https://github.com/egormkn/cpp-modern-development-course}{\color{color1} cpp-modern-development-course} --- решения задач специализации Coursera.
  \end{itemize}
}%
\en{
  My personal projects, courseworks and problem solutions can be found at \href{https://github.com/egormkn/}{\textbf{github.com/egormkn}}:
  \\
  \begin{itemize}
    \setlength{\itemindent}{2em}
    \item \href{https://github.com/egormkn/mbr-boot-manager}{\color{color1} mbr-boot-manager} --- x86 boot sector with partition selection menu,
    \item \href{https://github.com/egormkn/sdlxx}{\color{color1} SDLXX} --- С++ wrapper for C library SDL with a simple 2D scene manager,
    \item \href{https://github.com/egormkn/cpp-modern-development-course}{\color{color1} cpp-modern-development-course} --- solutions for the Coursera specialization.
  \end{itemize}
}

\cfoot{\small\textcolor{gray}{\href{https://github.com/egormkn/cv/releases}{\gitFirstTagDescribe\ (\getvariant)}}}

\end{document}
